\chapter*{ВВЕДЕНИЕ}
\addcontentsline{toc}{chapter}{ВВЕДЕНИЕ}

В 2023 году операционными системами на основе Linux пользуются 47\% разработчиков, 40\% веб--сайтов, 85\% смартфонов и 96\% серверов~\cite{stats}. Эти люди и устройства постоянно генерируют большое количество сетевого трафика, который нужно успевать обрабатывать.

Получением, отправкой и пересылкой трафика занимается сетевая подсистема Linux. Её мониторинг позволит выявить узкие места и правильно настроить её компоненты.

Целью данной курсовой работы --- разработка загружаемого модуля ядра, предоставляющего пользователю информацию о работе сетевой подсистемы Linux.

Для достижения поставленной в работе цели предстоит решить следующие задачи:
\begin{itemize}[label=---]
	\item произвести анализ функций и структур, используемых для обработки сетевых кадров;
	\item разработать загружаемый модуль ядра, предоставляющий информацию о работе сетевой подсистемы;
	\item реализовать загружаемый модуль ядра.
\end{itemize}
